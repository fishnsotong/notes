%% File              : notes.tex
%% Author            : Wayne Yeo [fishnsotong] <wwzyeo@gmail.com>
%% Date              : 2020-04-12T16:57:32+0800
%% Last Modified Date: 2020-04-16T02:06:42+0800
%% Last Modified By  : Wayne Yeo [fishnsotong] <wwzyeo@gmail.com>

\documentclass{article}

\usepackage{amsmath,amssymb,amsthm}
\usepackage[shortlabels]{enumitem}
\usepackage{physics}
\usepackage{titlesec}

\newtheorem{theorem}{Theorem}
\numberwithin{theorem}{section}
\newtheorem*{theorem*}{Theorem}
\newtheorem{corollary}{Corollary}
\numberwithin{corollary}{section}
\newtheorem*{corollary*}{Corollary}
\newtheorem{postulate}{Postulate}
\numberwithin{postulate}{section}
\newtheorem*{postulate*}{Postulate}
\newtheorem{lemma}{Lemma}
\numberwithin{lemma}{section}
\newtheorem*{lemma*}{Lemma}

\theoremstyle{definition}
\newtheorem{definition}{Definition}
\numberwithin{definition}{section}
\newtheorem*{definition*}{Definition}

\theoremstyle{remark}
\newtheorem*{remark}{Remark}

\titleformat
{\part} % command
[display] % shape
{\bfseries\Large\itshape} % format
{Lecture \ \thepart} % label
{0.5ex} % sep
{
    \rule{\textwidth}{1pt}
    \vspace{1ex}
    \centering
} % before-code
[
\vspace{-0.5ex}%
\rule{\textwidth}{0.3pt}
] % after-code

%% Preference for boldface vectors
\renewcommand{\vec}[1]{\mathbf{#1}}
\let\oldhat\hat
\renewcommand{\hat}[1]{\oldhat{\mathbf{#1}}}

\usepackage{hyperref}
\hypersetup{
    colorlinks=true,
    linkcolor=blue,
    filecolor=magenta,
    urlcolor=blue,
}

\urlstyle{same}

\DeclareMathOperator{\E}{\mathbb{E}}

\title{8.012 Lecture Notes}
\author{Wayne Yeo Wei Zhong}

\begin{document}

\maketitle

\begin{abstract}
  This class is an introduction to classical mechanics for students who are
  comfortable with calculus. Familiarity with vectors and differential equations
  are not assumed. The main topics are:
  vectors, kinematics, forces, motion, momentum, energy, angular motion, angular
  momentum, gravity, planetary motion, moving frames, and the motion of rigid
  bodies. The course follows the book \emph{An Introduction to Mechanics} by
  Kleppner and Kolenkow.
\end{abstract}
\tableofcontents
\newpage

\part{Vectors and Kinematics}

\setcounter{section}{0}

\section{Introduction to Classical Mechanics}

\subsection{Scope and limits of classical mechanics}

Classical mechanics is the study of \emph{motion} due to given forces, and aims
to establish some general rules regarding them. Some systems of interest
include: gravity, springs, pulleys, rigidity and friction.

\subsubsection{Extraordinary richness of phenomena}

\begin{itemize}
  \item Orbital mechanics: sensitivity to initial conditions, Venus' spin
    \href{https://www.nature.com/articles/35071000}{[10.1038/35071000]}
  \item Euler's Disk
  \item Rattleback
\end{itemize}

\subsubsection{Limits}

\begin{itemize}
  \item large velocities -- special relativity
  \item large distances, very strong gravity -- general relativity
  \item \underline{fields}, radiation
  \item very small distances, structure of matter -- quantum mechanics
\end{itemize}

\textbf{BUT} we can see that the following concepts that are present in other
fields of physics build on our understanding of them in classical mechanics.

\begin{itemize}
  \item $\vec{p}$ (momentum)
  \item $E$ (energy)
  \item $\vec{L}$ (angular momentum)
  \item symmetry
\end{itemize}

Careful definition of displacement, force and time are important. Correspondence
principle.

\section{Elementary concepts of classical mechanics}

\subsubsection{Model of space}

\subsubsection{Model of time}

\subsubsection{Model of matter}

To summarise the past three subsections, we can arrive at a particularly cogent
definition on the nature of classical mechanics.

\begin{definition*}
  \emph{Classical mechanics} is the study of mechanics concerning the motion of
  point masses in Euclidean geometric space under absolute, non-relativistic
  time.
\end{definition*}

\section{Vectors}

\subsection{Vector Algebra}

\begin{definition*}
  A \emph{vector} is a set of numbers accompanied by rules for how they change
  when the coordinate system is changed. For our purposes, we use a geometric
  definition: a vector is a \emph{directed line segment} with scale length and
  direction.
\end{definition*}

\subsubsection{Basic operations}

\paragraph{Unit vectors.} If the length of a vector is one unit, we call it a
unit vector. The vector of unit length parallel to $\vec{A}$ is $\hat{A}$. It
follows that \[
  \hat{A} = \frac{\vec{A}}{A}.
\]

Since we know that both $\vec{A}$ and $A$ have units of length, we can conclude
that $\hat{A}$ has units $L/L = 1$. The physical dimension of a vector is
carried by its magnitude, and unit vectors are dimensionless.

\paragraph{Scalar multiplication.} $\vec{C} = b\vec{A}$ If $b > 0$, $C$ is
parallel to $A$ and thus share the same unit vector.

If $b < 0$, then $\vec{C} = b\vec{A}$ is antiparallel to $\vec{A}$ 

\paragraph{Vector addition.} The geometrical interpretation shown in the book is
the tip-to-tail method.

\paragraph{Vector subtraction.} $\vec{A} - \vec{B} = \vec{A} + (- \vec{B})$

\paragraph{Algebraic properties of vectors.}

\subsubsection{Multiplying vectors}

\subsection{Rectangular coordinates}

The vector operations in the previous section were defined \emph{independently}
of any particular coordinate system. To invoke the concrete, physical nature of
analytic geometry, we now choose the rectangular (Cartesian) coordinate system
to work with for its simplicity and ease of visualization.

\subsubsection{Components of vectors}

\subsubsection{Base vectors}

\section{Kinematics}

\subsection{The position vector $\vec{r}$ and Displacement}

\subsection{Velocity and Acceleration}

\end{document}
