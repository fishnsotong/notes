%% File              : notes.tex
%% Author            : Wayne Yeo [fishnsotong] <wwzyeo@gmail.com>
%% Date              : 2020-04-12T16:57:32+0800
%% Last Modified Date: 2020-04-14T01:37:02+0800
%% Last Modified By  : Wayne Yeo [fishnsotong] <wwzyeo@gmail.com>

\documentclass{article}

\usepackage{amsmath,amssymb,amsthm}
\usepackage[shortlabels]{enumitem}
\usepackage{physics}
\usepackage{titlesec}

\newtheorem{theorem}{Theorem}
\numberwithin{theorem}{section}
\newtheorem*{theorem*}{Theorem}
\newtheorem{corollary}{Corollary}
\numberwithin{corollary}{section}
\newtheorem*{corollary*}{Corollary}
\newtheorem{postulate}{Postulate}
\numberwithin{postulate}{section}
\newtheorem*{postulate*}{Postulate}
\newtheorem{lemma}{Lemma}
\numberwithin{lemma}{section}
\newtheorem*{lemma*}{Lemma}

\theoremstyle{definition}
\newtheorem{definition}{Definition}
\numberwithin{definition}{section}
\newtheorem*{definition*}{Definition}

\titleformat
{\part} % command
[display] % shape
{\bfseries\Large\itshape} % format
{Lecture \ \thepart} % label
{0.5ex} % sep
{
    \rule{\textwidth}{1pt}
    \vspace{1ex}
    \centering
} % before-code
[
\vspace{-0.5ex}%
\rule{\textwidth}{0.3pt}
] % after-code

%% Preference for boldface vectors
\renewcommand{\vec}[1]{\mathbf{#1}}
\let\oldhat\hat
\renewcommand{\hat}[1]{\oldhat{\mathbf{#1}}}

\DeclareMathOperator{\E}{\mathbb{E}}

\title{8.012 Lecture Notes}
\author{Wayne Yeo Wei Zhong}

\begin{document}

\maketitle

\begin{abstract}
  This class is an introduction to classical mechanics for students who are
  comfortable with calculus. Familiarity with vectors and differential equations
  are not assumed. The main topics are:
  vectors, kinematics, forces, motion, momentum, energy, angular motion, angular
  momentum, gravity, planetary motion, moving frames, and the motion of rigid
  bodies. The course follows the book \emph{An Introduction to Mechanics} by
  Kleppner and Kolenkow.
\end{abstract}
\tableofcontents
\newpage

\part{Vectors and Kinematics}

\setcounter{section}{0}
\section{Vectors}

\subsection{Vector Algebra}

\begin{definition*}
  A \emph{vector} is a set of numbers accompanied by rules for how they change
  when the coordinate system is changed. For our purposes, we use a geometric
  definition: a vector is a \emph{directed line segment} with scale length and
  direction.
\end{definition*}

\subsubsection{Basic operations}

\paragraph{Unit vectors.} If the length of a vector is one unit, we call it a
unit vector. The vector of unit length parallel to $\vec{A}$ is $\hat{A}$. It
follows that \[
  \hat{A} = \frac{\vec{A}}{A}.
\]

Since we know that both $\vec{A}$ and $A$ have units of length, we can conclude
that $\hat{A}$ has units $L/L = 1$. The physical dimension of a vector is
carried by its magnitude, and unit vectors are dimensionless.

\paragraph{Scalar multiplication.} $\vec{C} = b\vec{A}$ If $b > 0$, $C$ is
parallel to $A$ and thus share the same unit vector.

If $b < 0$, then $\vec{C} = b\vec{A}$ is antiparallel to $\vec{A}$ 

\paragraph{Vector addition.} The geometrical interpretation shown in the book is
the tip-to-tail method.

\paragraph{Vector subtraction.} $\vec{A} - \vec{B} = \vec{A} + (- \vec{B})$

\paragraph{Algebraic properties of vectors.}

\subsubsection{Multiplying vectors}

\section{Kinematics}

\end{document}
