% File              : Physics.tex
% Author            : Wayne Yeo [fishnsotong@gosigs] <wwzyeo@gmail.com>
% Date              : 2020-06-09T01:08:56+0800
% Last Modified Date: 2020-06-09T01:09:02+0800
% Last Modified By  : Wayne Yeo [fishnsotong@gosigs] <wwzyeo@gmail.com>

\documentclass{article}

\usepackage{amsmath,amssymb}
\usepackage[margin=0.5in]{geometry}

\setlength{\parindent}{0pt}
\setlength{\parskip}{1.5pt}

\begin{document}

\section*{Physics}

\section{Measurement}

\paragraph{Base units.} \begin{tabular}{|l|l|l|}
\hline
\textbf{Name} & \textbf{Symbol} & \textbf{Measure} \\
\hline
metre & \text{m} & length \\
kilogram & \text{kg} & mass \\
second & \text{s} & time \\
ampere & \text{A} & electric current \\
kelvin & \text{K} & thermodynamic temperature \\
mole & \text{mol} & amount of substance \\
candela & \text{cd} & luminous intensity (not in syllabus) \\
\hline
\end{tabular}

\paragraph{Prefixes.} \begin{tabular}{|l|llllllllll|}
\hline
\textbf{Symbol} & p & n & $\mu$ & m & c & d & k & M & G & T \\
\hline
\textbf{Prefix} & pico & nano & micro & milli & centi & deci & kilo & mega & giga & tera \\
\textbf{Quantity} & $10^{-12}$ & $10^{-9}$ & $10^{-6}$ & $10^{-3}$ & $10^{-2}$ & $10^{-1}$ & $10^3$ & $10^6$ & $10^9$ & $10^{12}$ \\
\hline
\end{tabular}

\paragraph{Principle of Homogenity.} The units of quantities on both sides of an equation must be the same.

\paragraph{Systematic Errors.} Errors of measurements that occur according to some `fixed rule or pattern' such that they yield a consistent over-estimation or under-estimation of the true value.

\paragraph{Random Errors.} Errors with different magnitudes and signs in repeated measurements.

\paragraph{Precision.} Extent or limit of sensitivity of a given instrument to obtain the readings of the physical quantity being measured.

\paragraph{Accuracy.} How close a measured value is from the true value.

\paragraph{Consequential Uncertainties.} For \textbf{addition} and \textbf{subtraction}, $\Delta R = \Delta R_1 + \Delta R_2 + \ldots$. For \textbf{multiplication} and \textbf{division}, $\frac{\Delta R}{R} = \frac{\Delta R_1}{R_1} + \frac{\Delta R_2}{R_2} + \ldots$.

\section{Newtonian Mechanics}

\subsection{Kinematics}

\paragraph{Displacement ($s$).} Shortest distance from initial to final position of a body.

\paragraph{Velocity ($v$).} Rate of change of displacement ie. $v = \frac{\mathrm{d}s}{\mathrm{d}t}$.

\paragraph{Acceleration ($a$).} Rate of change of velocity ie. $a = \frac{\mathrm{d}v}{\mathrm{d}t} = \frac{\mathrm{d}^2s}{\mathrm{d}t^2}$.

\paragraph{Basic Equations of Kinematics.} Applicable only if $a$ is constant. Taking $u$ as initial velocity and $v$ as final velocity,

\begin{equation}
v = u + at
\end{equation}
\begin{equation}
s = \frac{1}{2}(u + v)t
\end{equation}
\begin{equation}
s = ut + \frac{1}{2}at^2
\end{equation}
\begin{equation}
v^2 = u^2 + 2as
\end{equation}

\subsection{Dynamics}

\paragraph{Inertia.} Reluctance of a body to change its state of rest or motion.

\paragraph{Linear momentum ($p$).} $p = mv$ where $m$ is the mass of the body and $v$ is the velocity of the body.

\paragraph{Newton's First Law.} Every body continues in its state of rest or uniform motion in a straight line unless an unbalanced force acts on it to change that state.

\paragraph{Newton's Second Law.} The rate of change of momentum of a body is proportional to the unbalanced force acting on it, and occurs in the direction of the force ie. $F \propto \frac{\mathrm{d}p}{\mathrm{d}t}$.

\paragraph{Newton's Third Law.} If a body exerts a force on another body, then that body will exert an equal and opposite force on the first body. Action-reaction forces must satisfy these conditions:

\begin{enumerate}
\item Equal in magnitude
\item Oppositely directed
\item Act on two different bodies
\item Be of the same type
\end{enumerate}

\paragraph{Impulse ($J$).} $J = \int^{t_2}_{t_1} \! F \, \mathrm{d}{t}$

\paragraph{Impulse-Momentum Theorem.} Impulse is equal to the change in momentum. Consequently, \begin{equation}
F_{av} = \frac{\Delta p}{\Delta t}
\end{equation}

\paragraph{Principle of Conservation of Momentum.} The total momentum of a system is conserved provided no external unbalanced force acts on the system ie. $p_\text{initial} = p_\text{final}$.

\paragraph{Elastic Collision}. Kinetic energy of a system is conserved.  Consequently, the relative velocity of approach is equal to the relative velocity of separation.

\paragraph{Inelastic Collision.} Refer to the definition of an elastic collision. For a totally inelastic collision, coalescence occurs.

\subsection{Forces}

\paragraph{Hooke's Law.} The force acted on by a spring is equal to \begin{equation}
F = -kx
\end{equation} where $k$ is the spring constant (N/m) and $x$ is the deformation (m). Correspondingly, the elastic potential energy stored in springs is $\frac{1}{2}kx^2$, or $\frac{F^2}{2k}$.

\paragraph{Hydrostatic Pressure ($p$).} Pressure acting on an object due to a fluid column above it. \begin{equation}
p = h \rho g
\end{equation} where $h$ is the height of the fluid column, $\rho$ is the density of fluid, and $g$ is the gravitational field strength.

\paragraph{Buoyancy/Upthrust ($U$).} Force acting upwards on an object submerged in a fluid. \begin{equation}
U = h \rho gA = m_\text{fluid}g
\end{equation} where $h$ is the height of the fluid column, $A$ is the cross-sectional area of the fluid column displaced, $\rho$ is the density of fluid displaced, and $m_\text{fluid}$ is the mass of the fluid column, and $g$ is the gravitational field strength.

\paragraph{Principle of Floatation.} A floating object will displace its own weight equivalent of the fluid it is floating on.

\paragraph{Friction.} Force that resists the motion of one surface relative to another with which it is in contact.

\paragraph{Drag/Viscous Force.} Internal friction in a fluid.

\paragraph{Torque ($\tau$).} Product of the force and the perpendicular distance from the line of action of the force to the axis.

Taking pivot about $A$ and a force applied at $B$, $\tau = rF\sin\theta$ where $\tau$ is the torque, $r$ is the distance $AB$ and $\theta$ is the angle between $AB$ and $F$.

\paragraph{Couple.} A pair of equal and oppositely-directed forces whose lines of actions do not coincide.

\paragraph{Equilibrium.} A body is said to be in equilibrium if these two conditions are met:

\begin{enumerate}
\item Translation equilibrium ie. $\sum F = 0$.
\item Rotational equilibrium ie. $\sum \tau = 0$.
\end{enumerate}

\subsection{Work, Energy and Power}

\paragraph{Work Done ($W$).} Product of the magnitude of the displacement and the component of the force parallel to the displacement. The work done by a force $F$ in moving an object by $\Delta x$ is \begin{equation}
W = F\cos\theta(\Delta x)
\end{equation} where $\theta$ is the angle between $F$ and $\Delta x$.

\paragraph{Kinetic Energy ($E_k$).} The kinetic energy of an object is \begin{equation}
E_k = \frac{1}{2}mv^2
\end{equation} where $m$ is the mass of the object

\paragraph{Gravitational Potential Energy ($U$).} (This is valid only near the Earth's surface.) The gravitational potential energy of an object is \begin{equation}
U = mgh
\end{equation} where $m$ is the mass of the object, $g$ is the gravitational field strength, and $h$ is the height of the object relative to the ground.

\paragraph{Power ($P$).} Rate at which work is done, or which energy is transferred. \begin{equation}
P = \frac{\mathrm{d}W}{\mathrm{d}t} = Fv
\end{equation}

\paragraph{Efficiency ($\gamma$).} Ratio of useful work performed by a machine or process. \begin{equation}
\gamma = \frac{P_w}{P_i} = \frac{E_w}{E_i}
\end{equation}

\subsection{Motion in a Circle}

\paragraph{Angular Displacement ($\theta$).} Angle through which the object moves on a circular path. \begin{equation}
\theta = \frac{s}{r}
\end{equation} where $s$ is the arc length and $r$ is the radius of the circular path.

\paragraph{Angular Velocity ($\omega$).} Rate of change of angular position of a rotating body. \begin{equation}
\omega = \frac{\mathrm{d}\theta}{\mathrm{d}t} = \frac{2\pi}{T} = 2\pi f
\end{equation}

\paragraph{Tangential Velocity ($v = r\omega$).} Component of velocity tangential to path of movement.

\paragraph{Centripetal Acceleration ($a$).} Acceleration of an object towards the centre of its circular path. \begin{equation}
a = v\omega = r\omega^2 = \frac{v^2}{r}
\end{equation}

\paragraph{Centripetal Force ($F$).} Force acting on an object towards the centre of its circular path. \begin{equation}
F = mr\omega^2 = m\frac{v^2}{r}
\end{equation}

\subsection{Gravitational Field}

\paragraph{Law of Universal Gravitation.} The gravitational force $F$ between two particles of masses $M$ and $m$ with distance $r$ apart is \begin{equation}
F = -\frac{GMm}{r^2}
\end{equation} where $G$ is the universal gravitational constant.

\paragraph{Gravitational Field Strength ($g$).} \textbf{Force per unit mass} acting on any mass placed at that point. \begin{equation}
g = \frac{F}{m} = -\frac{GM}{r^2}
\end{equation}

\paragraph{Gravitational Potential Energy ($U$).} \textbf{Work done} by an external agent in bringing a test mass from infinity to a point without any change in kinetic energy (taking infinity as the zero reference). \begin{equation}
F = -\frac{\mathrm{d}U}{\mathrm{d}r} \implies U = -\frac{GMm}{r}
\end{equation}

\paragraph{Gravitational Potential ($\Phi$).} \textbf{Work done per unit mass} by an external agent in bringing an object from infinity to a point without any change in kinetic energy (taking infinity as the zero reference). \begin{equation}
g = -\frac{\mathrm{d}\Phi}{\mathrm{d}r} \implies \Phi = -\frac{GM}{r}
\end{equation}

\paragraph{Kepler's Third Law.} For a planet of mass $m$ orbiting around a sun of mass $M$ with distance $r$, \begin{equation}
\frac{GMm}{r^2} = mr\omega^2 = mr\left(\frac{2\pi}{T}\right)^2 \implies T^2 = \frac{4\pi^2}{GM}r^3 \implies T^2 \propto r^3
\end{equation}

\section{Thermal Physics}

\subsection{Thermal Physics}

\paragraph{Internal Energy ($U = \mathrm{KE} + \mathrm{PE}$).} The sum of \textbf{molecular kinetic energy} and \textbf{intermolecular potential energy} of all the atoms of molecules of the system.

\paragraph{Zeroth Law of Thermodynamics.} If two bodies A and B are separately in thermal equilibrium with a third body C, then A and B are also in thermal equilibrium with each other.

\paragraph{Empirical Scale.} Scales based on observation and experimental measurement of thermometric properties as they change with temperature. (Usually, two fixed points are used.) Empirical scales have two issues: \begin{enumerate}
\item Linearity assumption leads to inaccuracy, as thermometric property may not vary linearly to temperature changes
\item Disagreement between different thermometers, as different thermometric properties vary in different ways to temperature changes
\end{enumerate}

\paragraph{Kelvin Scale.} Absolute zero (0 K) is used as the lower fixed point, and the triple point of water (273.16 K) is used as the upper fixed point.

\paragraph{Celsius Scale.} Kelvin scale shifted such that 0.01 $^{\circ}$C is the triple point of water ie. $\theta/^{\circ}\mathrm{C} = T/K - 273.15$.

\paragraph{Ideal Gas.} Perfect gas which obeys the combined gas law. Makes these assumptions:

\begin{enumerate}
\item \textbf{D}uration of impact among molecules and between molecules, and walls of container is negligible
\item \textbf{A}ttractive forces among molecules are negligible. Hence, molecules do not exert forces on each other except when they collide.
\item \textbf{V}olume of the atoms or molecules is negligible compared with the volume occupied by the gas.
\item \textbf{E}lastic collisions are present, so molecules bounce off the walls of the container and each other without losing any kinetic energy.
\end{enumerate}

\paragraph{Real Gas.} Real gases approach the characteristics of an ideal gas at low pressure and high temperature.

\paragraph{Combined Gas Law.} \begin{equation}
\frac{p_1V_1}{T_1} = \frac{p_2V_2}{T_2}
\end{equation} where $p$ is the pressure, $V$ is the volume occupied by the gas, and $T$ is the temperature of the gas.

\paragraph{Ideal Gas Law.} \begin{equation}
pV = nRT = NkT
\end{equation} where $n$ is number of moles, $R$ is the molar gas constant, $N$ is number of molecules, and $k$ is the Boltzmann constant.

\paragraph{Mean Translational Kinetic Energy.} \begin{equation}
E_k = \frac{1}{2}mv_{rms}^2 = \frac{3}{2}kT \implies v_{rms} = \sqrt{\frac{3kT}{m}}
\end{equation}

\subsection{First Law of Thermodynamics}

\paragraph{Heat Capacity ($C$).} Heat required to raise the temperature by 1 K. \begin{equation}
Q = C\Delta\theta
\end{equation}

\paragraph{Specific Heat Capacity ($c$).} Heat required to raise the temperature of 1 kg of the substance by 1 K. \begin{equation}
Q = mc\Delta\theta
\end{equation}

\paragraph{Specific Latent Heat of Fusion ($l_f$).} Heat required to change unit mass of the substance \textbf{from solid to liquid} state without a change in temperature. \begin{equation}
Q = ml_f
\end{equation}

\paragraph{Specific Latent Heat of Vaporization ($l_v$).} (Similar to specific latent heat of fusion, but applies to a change in state \textbf{from liquid to vapour}). \begin{equation}
Q = ml_v
\end{equation}

\paragraph{Kinetic Theory of Matter.} Macroscopic properties of matter:

\begin{tabular}{|l|p{5cm}|p{5cm}|p{5cm}|}
\hline
& \textbf{Distance apart} & \textbf{Motion} & \textbf{Internal energy} \\
\hline
\textbf{Solid} & Arranged close to each other in a regular pattern & Vibrations about their mean positions & Vibrational KE and PE \\
\textbf{Liquid} & Slightly further apart & Occurs in clusters and are free to slide around each other & Translational KE and PE \\
\textbf{Gas} & Very far apart from each other & Move in a random manner & Translational KE and PE \\
\hline
\end{tabular}

\paragraph{First Law of Thermodynamics.} The internal energy of a system depends only on its state, and the increase in internal energy of the system is the \textbf{sum} of the \textbf{heat supplied to the system} and the \textbf{external work done on the system} ie. $\Delta U = \Delta Q + \Delta W$.

\paragraph{Molar Heat Capacity at Constant Volume ($c_{v,m}$).} \begin{equation}
Q = nc_{v,m}\Delta T = \Delta U
\end{equation}

\paragraph{Molar Heat Capacity at Constant Pressure ($c_{p,m}$).} \begin{equation}
Q = nc_{p,m}\Delta T = \Delta U + p\Delta V
\end{equation}

\begin{tabular}{|l|l|l|l|l|}
\hline
\textbf{Process} & \textbf{Characteristic} & $\Delta U$ & $\Delta Q$ & $\Delta W$ \\
\hline
Isochoric & Constant volume. & $\Delta U = \Delta Q$ & $nc_{v,m}\Delta T$ & 0 \\
Isobaric & Constant pressure. & $\Delta U = \Delta Q + \Delta W$ & $nc_{p,m}\Delta T$ & -$p\Delta V$ \\
Isothermal & Constant temperature. & 0 & $-\Delta W$ & Area under graph. \\
Adiabatic & $\Delta Q = 0$ & $\Delta U = \Delta W$ & 0 & Area under graph. \\
Cyclic & $\Delta U = 0$ & 0 & $-\Delta W$ & Enclosed area. \\
\hline
\end{tabular}

\section{Oscillations and Waves}

\subsection{Oscillations}

Types of oscillations: \begin{itemize}
\item \textbf{Free oscillations} do not experience any resistive forces
\item \textbf{Damped oscillations} experience resistive forces that dissipate the energy of the system over time.
\item \textbf{Forced oscillations} are caused by the application of external periodic forces.
\end{itemize}

Considering the motion of an oscillating spring-mass system,

\paragraph{Equilibrium Position.} Position at which no net force acts on the oscillating mass.

\paragraph{Displacement ($x$).} Displacement of the oscillating mass from its equilibrium position.

\paragraph{Amplitude ($x_0$).} Maximum possible displacement attained by the oscillating mass from the equilibrium position.

\paragraph{Period ($T$).} Time taken for the system to undergo one complete oscillation.

\paragraph{Frequency ($f = \frac{1}{T}$).} Number of complete cycles per unit time made by the oscillating mass.

\paragraph{Angular Frequency ($\omega = 2\pi f$).}

\paragraph{Simple Harmonic Motion.} Vibratory motion in which the \textbf{acceleration} of the system is \textbf{proportional} to its \textbf{displacement from the equilibrium position}, and acts in the opposite direction to the displacement. \begin{equation}
a = -\omega^2 x \propto -x
\end{equation} where $\omega^2 = \frac{k}{m}$ for a spring-mass system. The solutions for simple harmonic motion are $x = x_0\sin\omega t$ or $x = x_0\cos\omega t$.

\paragraph{Velocity.} Taking $x = x_0\sin\omega t$, \begin{equation}
v = \frac{\mathrm{d}x}{\mathrm{d}t} = \omega x_0\cos\omega t \implies v_0 = \omega x_0
\end{equation}

Consequently, $v = \pm\omega\sqrt{x_0^2 - x^2}$.

\paragraph{Acceleration.} Taking $x = x_0\sin\omega t$, \begin{equation}
a = \frac{\mathrm{d}^2x}{\mathrm{d}t^2} = -\omega^2 x_0\sin\omega t \implies a_0 = \omega^2 x_0
\end{equation}

\paragraph{Kinetic Energy.} \begin{equation}
K = \frac{1}{2}m\omega^2 x_0^2 \cos^2 \omega t
\end{equation}

\paragraph{Potential Energy.} \begin{equation}
U = \frac{1}{2}kx^2 = \frac{1}{2} k (x_0\sin\omega t)^2 = \frac{1}{2}m\omega^2 x_0^2 \sin^2\omega t
\end{equation}

\paragraph{Total Energy.} \begin{equation}
E = \frac{1}{2}m\omega^2 x_0^2\cos^2\omega t + \frac{1}{2}m\omega^2 x_0^2\sin^2\omega t = \frac{1}{2}m\omega^2 x_0^2
\end{equation}

Types of damping: \begin{itemize}
\item \textbf{Critical damping.} The damping force is at an \textbf{exact amount} that allows the system to return to its equilibrium position \textbf{in the minimum possible time after it is displaced}.
\item \textbf{Light damping.} The damping force is \textbf{smaller} than that required for critical damping.
\item \textbf{Heavy damping.} The damping force is \textbf{larger} than that required for critical damping.
\end{itemize}

\paragraph{Resonance.} The state at which the \textbf{frequency of the external driving force} is \textbf{equal} to the \textbf{natural frequency of the oscillating system}. This results in the maximum transfer of energy from the driving system to the driven system, causing the system to respond with maximum amplitude.

\section{Waves}

\subsection{Wave Motion}

\paragraph{Wave.} Mechanism by which energy is transferred from one point to another.

\paragraph{Transverse Wave.} Direction of vibration of the wave particles is \textbf{perpendicular} to the direction of propagation of the wave.

\paragraph{Longitudinal Wave.} Direction of vibration of the wave particles is \textbf{parallel} to the direction of propagation of the wave.

\paragraph{Wavelength ($\lambda$).} Distance between two consecutive particles in the wave whose motions are in phase with one another.

\paragraph{Frequency ($f$).} Number of complete oscillations undergone by a wave particle per unit time.

\paragraph{Period ($T$).} Time taken for a wave particle to undergo one complete oscillation.

\paragraph{Amplitude ($A$).} Maximum displacement of a wave particle from its equilibrium position.

\paragraph{Velocity ($v$).} Velocity with which the wave profile advances. \begin{equation}
v = \frac{\lambda}{T} = \frac{\lambda}{1/f} \implies v = f\lambda
\end{equation}

\paragraph{Progressive Wave.} Wave in which the wave profile moves in the direction of propagation of the wave.

\paragraph{Phase Difference ($\Delta\phi$).} \begin{equation}
\Delta\phi = \frac{x}{\lambda} 2\pi = \frac{t}{T} 2\pi
\end{equation}

\paragraph{Intensity ($I = \frac{\text{Power}}{\text{Area}}$).} \begin{equation}
E = \frac{1}{2}m\omega^2 A^2 \implies I \propto A^2
\end{equation}

\paragraph{Polarisation.} A polarised wave oscillates in a single direction perpendicular to the direction of propagation of the wave.

\paragraph{Malus' Law.} For initial amplitude $A_0$ and transmitted amplitde $A_t$, considering initial intensity $I_0$ and transmitted intensity $I_t$, \begin{equation}
A_t = A_0\cos\theta \implies I_t = I_0\cos^2\theta
\end{equation} where $\theta$ is the angle that the vector of incident wave makes with the polarisation axis.

\paragraph{EM Wave Spectrum.} \begin{tabular}{|l|l|}
\hline
\textbf{EM Wave} & \textbf{Approximate Wavelength} \\
\hline
Gamma-rays & $<$ 0.1 nm \\
X-Ray & 1 nm to 10 nm \\
Ultraviolet & 10 nm to 100 nm \\
Visible & 400 nm (blue) to 700 nm (red) \\
Infra-Red & 1 $\mu$m to 1 mm \\
Microwave & 1 cm to 10 cm \\
Radio & $>$ 10 m \\
\hline
\end{tabular}

\subsection{Superposition}

\paragraph{Principle of Superposition.} The resultant wave displacement is given by the vectorial sum of the individual wave displacements at the point where the waves meet.

\paragraph{Stationary Wave.} Two waves of the same propagation speed, wavelength and amplitude overlap while travelling in opposite directions.

\paragraph{Fundamental Frequency.} Lowest resonant frequency of a vibrating object.

\paragraph{Harmonic.} Integer multiple of the fundamental frequency of a vibrating object.

\paragraph{Overtone.} Any resonant frequency above the fundamental frequency (an overtone may not be a harmonic).

\paragraph{String.} $L = \frac{n}{2} \lambda_n$ for \textit{n}th harmonic (both ends are nodes).

\paragraph{Closed Pipes.} $L = \frac{2n - 1}{4} \lambda_n$ for \textit{n}th harmonic (open end is antinode, closed end is node, $n$ must be odd).

\paragraph{Open Pipes.} $L = \frac{n}{2} \lambda_n$ for \textit{n}th harmonic (both ends are antinodes).

\paragraph{End Correction.} The distance of the anti-node away from the open end.

\paragraph{Diffraction.} Spreading of waves when they encounter an aperture or obstacle whose linear dimension is comparable to the wavelength of the waves. Generally, the smaller the width of the aperture in relation to the wavelength, the greater is the spreading or diffraction of the waves.

\paragraph{Interference.} Phenomenon of two or more waves of the same type meeting at a point in space to product a resultant wave disturbance given by the superposition of individual waves at that point.

\paragraph{Conditions for steady-observable interference patterns:} \begin{itemize}
\item Coherence (constant phrase difference)
\item \textit{(Transverse waves)} Unpolarized/Polarized in the same plane
\item Roughly same amplitude (if the amplitude of one wave is much greater than the other, it is very difficult to observe interference)
\end{itemize}

Path and phase difference is denoted as $\Delta x$ and $\Delta \phi$ respectively.

\paragraph{Constructive Interference.} For two waves arriving at a point, $\Delta x = n\lambda$ where $\lambda$ is the wavelength of the two waves, or $\Delta\phi = 2n\pi$.

\paragraph{Destructive Interference.} For two waves arriving at a point, $\Delta x = \left(n + \frac{1}{2}\right)\lambda$ where $\lambda$ is the wavelength of the two waves, or $\Delta\phi = (2n + 1)\pi$.

\paragraph{Young's Double Slit Experiment.} \begin{equation}
\text{Distance between two bright fringes, } \Delta y = \frac{\lambda D}{d}
\end{equation} where $D$ is the distance between the screen and double slits and $d$ is the distance between the two slits.

\paragraph{Rayleigh's Criterion.} The limiting angle of resolution, $\theta_{\text{min}}$, is determined by \begin{equation}
\sin\theta_{\text{min}} = \frac{\lambda}{a}
\end{equation} where $\lambda$ is the wavelength and $a$ is the width of the slit. For $\lambda << a$, $\sin\theta_{\text{min}} \approx \theta_{\text{min}}$ and $\theta_{\text{min}} \approx \frac{\lambda}{a}$.

\paragraph{Single Slit Diffraction Pattern.} For the \textbf{first minimum}, \begin{equation}
\sin\theta = \frac{\lambda}{a}
\end{equation} where $\theta$ is the angle of deviation, $\lambda$ is the wavelength of light used, and $a$ is the width of the slit.

\paragraph{Multiple Slit Diffraction Pattern.} For the \textbf{$\mathbf{n}$th order maximum}, \begin{equation}
d\sin\theta = n\lambda
\end{equation} where $d$ is the spacing of slits in the diffraction grating, $\theta$ is the angle of deviation, and $\lambda$ is the wavelength of light used.

\section{Electricity and Magnetism}

\subsection{Electric Fields}

Take $\epsilon$ to be the permittivity of the medium and $r$ to be the distance between two point charges.

\paragraph{Coulomb's Law.} The \textbf{force} acting on a point charge $Q_1$ on another point charge $Q_2$ is \begin{equation}
F = \frac{Q_1Q_2}{4\pi\epsilon r^2}
\end{equation}.

\paragraph{Electric Field Strength ($E$).} \textbf{Force per unit positive charge} placed at that point due to a point charge $Q$. \begin{equation}
E = \frac{F}{q} = \frac{Q}{4\pi\epsilon r^2}
\end{equation}

\paragraph{Electric Potential Energy ($U$).} The \textbf{potential energy} $U$ between two point charges $Q_1$ and $Q_2$ is \begin{equation}
F = -\frac{\mathrm{d}U}{\mathrm{d}r} \implies U = \frac{Q_1Q_2}{4\pi\epsilon r}
\end{equation}

\paragraph{Electric Potential ($V$).} \textbf{Work done per unit charge} by an external agent in bringing a small test charge from infinity to a point without change in kinetic energy. \begin{equation}
E = -\frac{\mathrm{d}V}{\mathrm{d}r} \implies V = \frac{Q}{4\pi\epsilon r}
\end{equation}

For two parallel metal plates with potential difference $V$, there is an uniform electric field $E = \frac{V}{d}$ where $d$ is the distance between the two plates.

\subsection{Current of Electricity}

\paragraph{Current ($I$).} Rate of flow of electric charge through a given cross-section of a conductor ie. $I = \frac{\mathrm{d}{Q}}{\mathrm{d}t}$. \begin{equation}
I = nAve
\end{equation} where $n$ is the number density of electrons, $A$ is the cross-sectional area of the conductor, $v$ is the average drift velocity of the electrons and $e$ is the elementary charge.

\paragraph{Charge ($Q$).} The charge which flows past a given cross section is the product of the steady current that flows past the section and the time during which the current flows ie. $Q = It$.

\paragraph{Coulomb.} Amount of electrical charge that passes through a given cross section of a circuit when a steady current of one ampere flows in one second.

\paragraph{Potential Difference ($V$).} Amount of electrical energy converted to other forms of energy when unit charge passes from one point to the other. \begin{equation}
V = \frac{W}{Q} = \frac{P}{I}
\end{equation}

\paragraph{Volt.} Potential difference between two points in a circuit if one joule of electrical energy is converted to other forms of energy when one coulomb of charge passes from one point to the other.

\paragraph{Resistance ($R$).} Ratio of potential difference across the conductor to the current flowing through it ie. $R = \frac{V}{I}$. The resistance across a conductor can be calculated by the following formula: \begin{equation}
R = \frac{\rho L}{A}
\end{equation} where $\rho$ is resistivity of the material, $L$ is the length of the conductor, and $A$ is the cross-sectional area of the conductor.

\paragraph{Ohm.} Resistance of a conductor if a current of one ampere flows in it when a potential difference of one volt is applied across it.

\paragraph{Internal Resistance.} Resistance of a cell due to chemicals in the cell.

\subsection{D.C. Circuits}

\paragraph{Kirchhoff's First Law (Current Law).} The algebraic sum of the currents at a junction of a circuit is zero ie. $\sum I = 0$. (This law is a consequence of the law of conservation of charge.)

\paragraph{Kirchhoff's Second Law (Voltage Law).} The algebraic sum of all electric potential changes around any closed loop is zero ie. $\sum \Delta V = 0$. (This law is a consequence of the law of conservation of energy.)

\paragraph{Effective Resistance of Resistors in Series.} \begin{equation}
R_{\text{eff}} = R_1 + R_2 + \ldots + R_n
\end{equation}

\paragraph{Effective Resistance of Resistors in Parallel.} \begin{equation}
\frac{1}{R_{\text{eff}}} = \frac{1}{R_1} + \frac{1}{R_2} + \ldots + \frac{1}{R_n}
\end{equation}

\paragraph{Electromotive Force.} Amount of energy converted from non-electrical to electrical form per unit charge that moves from one terminal of the cell to the other outside of the cell.

\paragraph{Potential-Divider Principle.} Consider a circuit with two resistors $R_1$ and $R_2$ connected in series, with a voltage $V$ applied across them. The voltage across $R_1$ is $V_1 = \frac{R_1}{R_1 + R_2}V$ and similarly, the voltage across $R_2$ is $V_2 = \frac{R_2}{R_1 + R_2}V$. (The rule applies to continuous wire resistors too.)

\paragraph{Thermistor.} Resistor whose electrical resistance varies with temperature. (Most commonly, the resistance of a thermistor will fall as its temperature rises.)

\paragraph{Light-Dependent Resistor (LDR).} Resistor whose electrical resistance decreases as the intensity (brightness) of the light falling onti it increases. (Also known as a photoresistor or photocell.)

\paragraph{Ground.} A conductor connected to the earth. The absolute potential of the ground, and the earth, is taken to be zero.

\subsection{Electromagnetism}

\paragraph{Magnetic Flux Density ($B$).} Force acting per unit length of a conductor which carries unit current and is at right angles to the magnetic field. \begin{equation}
F = BIL\sin\theta \implies B = \frac{F}{IL\sin\theta}
\end{equation} where $F$ is the force acting on the conductor, $I$ is the current passing through the conductor and $\theta$ is the angle that the conductor makes with the magnetic field.

\paragraph{Tesla.} Unit of magnetic flux density equivalent to a force of 1 N experience by a straight conductor of length 1 m and carrying a current of 1 A when it is placed perpendicular to a magnetic field.

\paragraph{Torque on a Current-Carrying Rectangular Loop.} For a loop with $N$ turns in a magnetic field $B$, the total torque on its vertical arms is \begin{equation}
\tau = BIAN
\end{equation} where $I$ is the current passing through the loop and $A$ is the area of the rectangular loop.

\paragraph{Force on a Charged Particle.} The force experienced by the charged particle in a magnetic field $B$ is given by \begin{equation}
F = Bqv\sin\theta
\end{equation} where $q$ is the charge of the particle, $v$ is the velocity of the particle, and $\theta$ is the angle between the magnetic field and the velocity vector.

\paragraph{Magnetic Field due to a Long, Straight Wire.} For an infinitely long, straight wire with current $I$, the magnetic field generated is \begin{equation}
B = \frac{\mu_0I}{2\pi r} \propto \frac{I}{r}
\end{equation} where $r$ is the distance away from the wire.

\paragraph{Magnetic Field due to a Solenoid.} For a solenoid with current $I$, the magnetic field generated at the centre is \begin{equation}
B = \mu_0nI \propto nI
\end{equation} where $n = \frac{N}{L}$, $N$ representing the number of turns and $L$ representing the length of the solenoid. The magnetic field of the solenoid at its ends is halved ie. $B = \frac{1}{2}\mu_0nI$.

\paragraph{Magnetic Field due to a Circular Coil.} For a circular coil of $N$ turns and radius $r$ with current $I$ passing through, the magnetic field generated at the centre is \begin{equation}
B = \frac{\mu_0NI}{2r} \propto \frac{NI}{r}
\end{equation}

\subsection{Electromagnetic Induction}

\paragraph{Magnetic Flux ($\phi$).} Product of the component of the magnetic field normal to the plane of the surface and the area of the surface ie. $\phi = BA\cos\theta$ where $\theta$ is the angle between the magnetic field and the normal to the area of the surface.

\paragraph{Total Flux Linkage ($\Phi$).} The total flux linkage is given by $\Phi = N\phi = NBA\cos\theta$.

\paragraph{Faraday's Law.} When the magnetic flux linkage with a circuit is changed, an induced emf is set-up whose magnitude is proportional to the rate of change of flux linkage ie. $\epsilon = \left|\frac{\mathrm{d}\phi}{\mathrm{d}t}\right|$.

\paragraph{Lenz's Law.} The direction of induced electromotive force is so directed as to oppose the change in flux causing it.

\paragraph{Electromagnetic Force Induced in a Long, Straight Wire.} $\epsilon = BLv$ where $L$ is the length of the wire and $v$ is the velocity of the wire.

\subsection{Alternating Currents}

\paragraph{Root-mean-square Current ($I_{rms}$).} \begin{equation}
I_{rms} = \sqrt{\langle I^2 \rangle} = \sqrt{\frac{1}{T} \int^{t_2}_{t_1} \! I^2 \, \mathrm{d}t}
\end{equation}

\paragraph{Average Power ($\langle P \rangle$).} \begin{equation}
\langle P \rangle = \langle I^2 \rangle R = I^2_{rms}R
\end{equation}

\paragraph{Sinusoidal A.C ($I = I_0\sin{\omega t}$).} \begin{equation}
I_{rms} = \frac{I_0}{\sqrt{2}}, \langle P \rangle = \frac{P_0}{2}
\end{equation}

\paragraph{Simple Iron-core Transformer.} \begin{equation}
\frac{N_s}{N_p} = \frac{V_s}{V_p} = \frac{I_p}{I_s}
\end{equation}

\paragraph{Half-Wave Rectification.} The use of a single diode in a circuit with an A.C. source converts the A.C. into a D.C. (However, as current can only pass through the diode in a single direction, half of the power is lost.)

\section{Modern Physics}

\subsection{Quantum Physics}

\paragraph{Photon Energy ($E$).} \begin{equation}
E = hf
\end{equation} where $h$ is Planck's constant and $f$ is the frequency of the electromagnetic radiation.

\paragraph{Work Function ($\phi$).} Minimum energy needed to remove an electron from the surface of a solid.

\paragraph{Photoelectric Equation.} When photons of frequency $f$ are shone onto a metal surface, \begin{equation}
hf = \frac{1}{2}m_ev_{\text{max}}^2 + \phi
\end{equation} where $v_{\text{max}}$ is the maximum velocity of photoelectrons emitted. Consequently, $\phi = hf_0$ where $f_0$ is the threshold frequency required for emission of photoelectrons.

\paragraph{Stopping Potential ($V_s$).} The electric potential required to stop emitted photoelectrons is \begin{equation}
eV_s = \frac{1}{2}m_ev_{\text{max}}^2 \implies V_s = \frac{hf - \phi}{e} = \frac{h(f - f_0)}{e}
\end{equation}

\paragraph{de Broglie Wavelength.} The wavelength of a particle is given by \begin{equation}
\lambda = \frac{h}{p} = \frac{h}{mv}
\end{equation} where $p$ is the momentum of the object.

\paragraph{Ground State.} Condition in which every electron is in the lowest energy state available.

\paragraph{Excitation Energy.} Energy required to raise the atom from its ground state to an excited state.

\paragraph{Ionisation Energy.} Energy required to raise an electron from level $n = 1$ to $n = \infty$.

\paragraph{Lyman, Balmer and Paschen Series.} The Lyman series refers to the downward transitions of electrons from any energy level in a hydrogen atom to the ground level $n = 1$. Similarly, the Balmer and Paschen series refer to the donward transitions of electrons to $n = 2$ and $n = 3$ respectively.

\paragraph{Line Spectra.} Refer to notes for details on emission line spectra and absorption line spectra.

\paragraph{X-Ray Spectrum.} Refer to notes for details on the line spectrum and continuous spectrum.

\paragraph{Bremsstrahlung Radiation (Braking Radiation).} Electromagnetic radiation produced when electrons in the X-ray tube are decelerated due to the electric and magnetic field of an atomic nucleus.

\paragraph{Heisenberg's Uncertainty Principle.} The uncertainties in the measurements of position and momentum, $\Delta x$ and $\Delta p$ respectively, are determined by \begin{equation}
\Delta x \Delta p \geq \frac{\hslash}{2} = \frac{h}{4\pi}
\end{equation} where $\hslash$ is the reduced Planck's constant.

\subsection{Nuclear Physics}

\paragraph{Rutherford's $\alpha$-particle Scattering Experiment.} Shows the existence and small size of the atomic nucleus. (Refer to notes for more details.)

\paragraph{Proton/Atomic Number ($Z$).} Number of protons within an atomic nucleus.

\paragraph{Nucleon/Mass Number ($A$).} Total number of nucleons (protons and neutrons) in an atomic nucleus.

\paragraph{Isotopes.} Atoms of the same element having the same atomic number $Z$ but different mass number $A$.

\paragraph{Relative Atomic Mass ($A_r$).} Mass of an atom by comparing its mass to that of the neutral carbon-12 isotope.

\paragraph{Unified Atomic Mass Constant ($u$).} The unified mass unit is defined such that the mass of an atom of the carbon-12 isotope is exactly $12u$.

\paragraph{Mass-Energy Equivalence.} Mass $m$ and energy $E$ are equivalent, by the formula $E = mc^2$ where $c$ is the speed of light.

\paragraph{Mass Defect.} Discrepancy in mass between the mass of nucleus and the mass of its nucleons.

\paragraph{Binding Energy.} Mass which disappears corresponding to an amount of energy released to form the stable nucleus from its component nucleons. (alt. Total amount of energy needed to split a stable nucleus into its free constituent nucleons.)

\paragraph{Binding Energy per Nucleon.} Measure of average amount of energy needed to extract a nucleon from the nucleous.

\paragraph{Conservation Laws in Nuclear Reactions.} There are three conservation laws: \begin{enumerate}
\item \textbf{Conservation of mass numbers.} The sum of the mass numbers of the nuclides before the reaction is the same as that of the nuclides after the reaction.
\item \textbf{Conservation of atomic numbers.} Charge is conserved in nuclear reactions. The sum of the atomic numbers of the nuclides before the reaction is the same as that of the nuclides after the reaction.
\item \textbf{Conservation of mass-energy.} The total mass-energy before and after the reaction must be the same. 
\end{enumerate}

\paragraph{Nuclear Fission.} When bombarded with a neutron, a heavy nucleus undergoes nuclear fission causing it to disintegrate into two lighter nuclei, which subsequently emits multiple neutrons.

\paragraph{Nuclear Fusion.} The combination of two light nuclei to form a heavier nucleus.

\paragraph{Radioactivity.} Spontaneous and random disintegration of a heavy unstable nucleus into a more stable nucleus through the emission of radiation such as $\alpha$-particles, $\beta$-particles and $\gamma$-rays.

Spontaneity refers to radioactive decays occuring of their own accord and being unaffected by physical and chemical factors.

Randomness refers to the impossibility of predicting exactly which and when a particular nucleus will disintegrate. Particles are not emitted at regular intervals of time, and it is impossible to know the direction of emission of particles.

\paragraph{Alpha Decay.} Alpha particles (namely, the helium nucleus ie. 2 protons and 2 neutrons bound together) are emitted from nuclei that are too large to be stable, with speed roughly $0.1c$ and forming roughly $10^4$ ion pairs per mm in air.

\paragraph{Beta Decay.} Beta particles (electrons or positrons) are emitted from nuclei that have a skewed neutron-to-proton ratio, with a speed of roughly $0.9c$ and forming roughly $10^2$ ion pairs per mm in air.

If a nucleus has too many neutrons, a neutron is converted into a proton, an electron, and an antineutrino.

If a nucleus has too many protons, a proton is converted into a neutron, a positron, and a neutrino.

\paragraph{Gamma Emission.} Nucleus resulting from alpha or beta decay are left in an excited energy state and may give up excess energy by emitting electromagnetic radiation, in the form of gamma rays. They form roughly one ion pair per mm in air.

\paragraph{Activity ($A$).} The activity of a radioactive material is the number of disintegrations of its atoms per unit time ie. $A = -\frac{\mathrm{d}N}{\mathrm{d}t}$.

\paragraph{Decay Constant ($\lambda$).} Probability of a particular atom decaying per unit time. Since $-\frac{\mathrm{d}N}{\mathrm{d}t} \propto N$, activity can be expressed as $A = \lambda N$. Likewise, $N = N_0e^{-\lambda t}$ and similarly for $A$, $m$ (mass) and $C$ (count rate).

\paragraph{Half Life ($t_{1/2}$).} Time taken for half the number of nuclei of a radioactive element to decay. \begin{equation}
t_{1/2} = \frac{\ln 2}{\lambda}
\end{equation}

\end{document}
