%% File              : notes.tex
%% Author            : Wayne Yeo [fishnsotong] <wwzyeo@gmail.com>
%% Date              : 2020-04-12T16:57:32+0800
%% Last Modified Date: 2020-08-26T15:48:07+0800
%% Last Modified By  : Wayne Yeo [fishnsotong@superheated] <wwzyeo@gmail.com>

\documentclass{article}

\usepackage{amsmath,amssymb,amsthm}
\usepackage[shortlabels]{enumitem}
\usepackage{physics} \usepackage{titlesec}
\usepackage{achemso}
\usepackage{chemmacros}
  \chemsetup{modules={all}}

\usepackage{epigraph}
\setlength\epigraphwidth{10cm}

\hyphenpenalty=1000

\newtheorem{theorem}{Theorem}
\numberwithin{theorem}{section}
\newtheorem*{theorem*}{Theorem}
\newtheorem{corollary}{Corollary}
\numberwithin{corollary}{section}
\newtheorem*{corollary*}{Corollary}
\newtheorem{postulate}{Postulate}
\numberwithin{postulate}{section}
\newtheorem*{postulate*}{Postulate}
\newtheorem{lemma}{Lemma}
\numberwithin{lemma}{section}
\newtheorem*{lemma*}{Lemma}
\newtheorem{definition}{Definition}
\numberwithin{definition}{section}
\newtheorem*{definition*}{Definition}
\newenvironment{justification} {\begin{proof}[Justification]} {\end{proof}}


\titleformat
{\part} % command
[display] % shape
{\bfseries\Large\itshape} % format
{Part \ \thepart} % label
{0.5ex} % sep
{
    \rule{\textwidth}{1pt}
    \vspace{1ex}
    \centering
} % before-code
[
\vspace{-0.5ex}%
\rule{\textwidth}{0.3pt}
] % after-code

\DeclareMathOperator{\E}{\mathbb{E}}

\title{Non-equilibrium systems and molecular machines}
\author{Wayne Yeo Wei Zhong}

\begin{document}

\maketitle

\begin{abstract}
  This course deals primarily with non-equilibrium properties of familiar
  chemical systems, and builds a theoretical foundation explaining the frontier
  topics of molecular switches, motors and replicators.
\end{abstract}

\tableofcontents

\newpage

\section*{\centering{Useful Relations}}
\bigskip

\begin{equation*}
  \Delta_r G^{\standardstate} = -RT\ln{K}
\end{equation*}

\newpage

\part{Non-equilibrium systems}

\epigraph{[\textbf{Thermodynamics} is] the only physical theory of universal
content concerning which I am convinced that, within the framework of the
applicability of its basic concepts, it will never be overthrown.}{Einstein}

The primary objective of chemical thermodynamics is the establishment of a
criterion to determine the feasibility or spontaneity of a given
transformation. Some examples of processes that we consider are chemical
reactions, phase changes and the formation of solutions. 

The thermodynamic description of these chemical systems employ the \textbf{laws
of thermodynamics}, that form an axiomatic basis. The first law specifies that
energy can be exchanged between physical systems as heat and work.
The second law defines the existence of a quantity called entropy, that
describes the thermodynamic direction in which the system can evolve.

In thermodynamics, we make certain distinctions about what the system is. The
\textbf{thermodynamic system} is a precisely defined region of the universe
under study. Everything in the universe except the system is called the
surroundings.

In conclusion, in thermodynamics we consider whether these processes \textit{can
happen}, and \textit{to what extent} they happen. It is important to note that
thermodynamics remains silent regarding \textit{how fast} they occur, which will
be dealt with in kinetics at a later stage.

\subsection{Further reading}

\subsubsection{Books}
At the moment, I'm trying to decide upon whether Atkins or McQuarrie gives a
better treatment on the subject. Perhaps, by the time I'm done with these notes,
I'll be better prepared to say.

\begin{itemize}

  \item James Keeler and Peter Wothers \textit{Why Chemical Reactions Happen},
    Oxford University Press, 2003: a qualitative, accessible explanation of the
    motivating ideas

  \item Peter Atkins and Julio de Paula \textit{Physical Chemistry}, Oxford
    University Press
\end{itemize}

\setcounter{section}{0}
\section{States, equations of state, 0th law}

\subsection{Laws of thermodynamics}

\paragraph{0th law of thermodynamics.} If $A$ and $B$ are in thermal equilibrium
and $B$ and $C$ are in thermal equilibrium, then $A$ and $C$ are in thermal 
equilibrium. \textit{Defines Kelvin scale.}

\paragraph{1st law of thermodynamics.} $U$ is conserved. \textit{You can break
even.}

\begin{equation}
  \mathrm{d}w + \mathrm{d}q = \mathrm{d}U
\end{equation}

\paragraph{2nd law of thermodynamics.} Defines entropy and the direction of
time. \textit{You can break even at 0 K}.

\begin{equation}
  \Delta S = \int \frac{\textnormal{d} q_{\mathrm{rev}}}{T}
\end{equation}

\paragraph{3rd law of thermodynamics.} Defines the absolute scale of $T$. For
pure crystal, $S \rightarrow 0$ as $T \rightarrow 0$. \textit{You can never
reach 0 K}.

\part{Molecular machines}

\end{document}
