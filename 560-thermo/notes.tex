%% File              : notes.tex
%% Author            : Wayne Yeo [fishnsotong] <wwzyeo@gmail.com>
%% Date              : 2020-04-12T16:57:32+0800
%% Last Modified Date: 2020-08-26T15:34:04+0800
%% Last Modified By  : Wayne Yeo [fishnsotong@superheated] <wwzyeo@gmail.com>

\documentclass{article}

\usepackage{amsmath,amssymb,amsthm}
\usepackage[shortlabels]{enumitem}
\usepackage{physics} \usepackage{titlesec}
\usepackage{achemso}
\usepackage{chemmacros}
  \chemsetup{modules={all}}

\usepackage{epigraph}
\setlength\epigraphwidth{10cm}

\hyphenpenalty=1000

\newtheorem{theorem}{Theorem}
\numberwithin{theorem}{section}
\newtheorem*{theorem*}{Theorem}
\newtheorem{corollary}{Corollary}
\numberwithin{corollary}{section}
\newtheorem*{corollary*}{Corollary}
\newtheorem{postulate}{Postulate}
\numberwithin{postulate}{section}
\newtheorem*{postulate*}{Postulate}
\newtheorem{lemma}{Lemma}
\numberwithin{lemma}{section}
\newtheorem*{lemma*}{Lemma}
\newtheorem{definition}{Definition}
\numberwithin{definition}{section}
\newtheorem*{definition*}{Definition}
\newenvironment{justification} {\begin{proof}[Justification]} {\end{proof}}


\titleformat
{\part} % command
[display] % shape
{\bfseries\Large\itshape} % format
{Part \ \thepart} % label
{0.5ex} % sep
{
    \rule{\textwidth}{1pt}
    \vspace{1ex}
    \centering
} % before-code
[
\vspace{-0.5ex}%
\rule{\textwidth}{0.3pt}
] % after-code

\DeclareMathOperator{\E}{\mathbb{E}}

\title{5.60 Thermodynamics and Kinetics}
\author{Wayne Yeo Wei Zhong}

\begin{document}

\maketitle

\begin{abstract}
  This course deals primarily with equilibrium properties of macroscopic
  systems, basic thermodynamics, chemical equilibirum of reactions in gas and
  solution phase, and rates of chemical reactions. Additional topics include
  electrochemistry, photochemistry, catalysis and enzyme kinetics.
\end{abstract}

\tableofcontents

\newpage

\section*{\centering{Useful Relations}}
\bigskip

\begin{equation*}
  \Delta_r G^{\standardstate} = -RT\ln{K}
\end{equation*}

\begin{equation*}
\Delta_r G^{\standardstate} = \Delta_r H^{\standardstate} - T \Delta_r S^{\standardstate}
\end{equation*}

\begin{equation*}
  E = E^{\standardstate'} + \frac{RT}{n'F} \ln{\frac{C_\mathrm{P}}{C_\mathrm{Q}}}
\end{equation*}
\newpage

\part{Thermodynamics}

\epigraph{[\textbf{Thermodynamics} is] the only physical theory of universal
content concerning which I am convinced that, within the framework of the
applicability of its basic concepts, it will never be overthrown.}{Einstein}

The primary objective of chemical thermodynamics is the establishment of a
criterion to determine the feasibility or spontaneity of a given
transformation. Some examples of processes that we consider are chemical
reactions, phase changes and the formation of solutions. 

The thermodynamic description of these chemical systems employ the \textbf{laws
of thermodynamics}, that form an axiomatic basis. The first law specifies that
energy can be exchanged between physical systems as heat and work.
The second law defines the existence of a quantity called entropy, that
describes the thermodynamic direction in which the system can evolve.

In thermodynamics, we make certain distinctions about what the system is. The
\textbf{thermodynamic system} is a precisely defined region of the universe
under study. Everything in the universe except the system is called the
surroundings.

In conclusion, in thermodynamics we consider whether these processes \textit{can
happen}, and \textit{to what extent} they happen. It is important to note that
thermodynamics remains silent regarding \textit{how fast} they occur, which will
be dealt with in kinetics at a later stage.

\subsection{Further reading}

\subsubsection{Books}
At the moment, I'm trying to decide upon whether Atkins or McQuarrie gives a
better treatment on the subject. Perhaps, by the time I'm done with these notes,
I'll be better prepared to say.

\begin{itemize}

  \item James Keeler and Peter Wothers \textit{Why Chemical Reactions Happen},
    Oxford University Press, 2003: a qualitative, accessible explanation of the
    motivating ideas

  \item Peter Atkins and Julio de Paula \textit{Physical Chemistry}, Oxford
    University Press
\end{itemize}

\setcounter{section}{0}
\section{States, equations of state, 0th law}

\subsection{Laws of thermodynamics}

\paragraph{0th law of thermodynamics.} If $A$ and $B$ are in thermal equilibrium
and $B$ and $C$ are in thermal equilibrium, then $A$ and $C$ are in thermal 
equilibrium. \textit{Defines Kelvin scale.}

\paragraph{1st law of thermodynamics.} $U$ is conserved. \textit{You can break
even.}

\begin{equation}
  \mathrm{d}w + \mathrm{d}q = \mathrm{d}U
\end{equation}

\paragraph{2nd law of thermodynamics.} Defines entropy and the direction of
time. \textit{You can break even at 0 K}.

\begin{equation}
  \Delta S = \int \frac{\textnormal{d} q_{\mathrm{rev}}}{T}
\end{equation}

\paragraph{3rd law of thermodynamics.} Defines the absolute scale of $T$. For
pure crystal, $S \rightarrow 0$ as $T \rightarrow 0$. \textit{You can never
reach 0 K}.

\subsection{Definitions}

\subsubsection{Types of systems}
\begin{itemize}
  \item Open systems (free transfer of energy and matter).
  \item Closed systems (free transfer of energy, but not matter).
  \item Isolated systems (no transfer of energy and matter).
\end{itemize}

\subsubsection{Types of properties}
\begin{itemize}
  \item{Intensive properties (temperature, molar volume, molar mass, density)}
  \item{Extensive properties (mass, volume, pressure)}
\end{itemize}

To get an intensive property from an extensive property, divide the property by
its molar amount.

\subsection{Properties of gases}

\paragraph{Ideal gas law. }
\begin{equation}
  pV = nRT
\end{equation}

\paragraph{Van der Waals equation of state.} Extension of the ideal gas equation
to account for intermolecular interactions, where $\overline{V}$ denotes molar
volume. 

\begin{equation}
  \left( p + \frac{a}{\overline{V}^2} \right) (\overline{V} - b) = RT 
\end{equation}

Notice that Equation 4 reduces to $pV = nRT$ when $\overline{V}$ is large.

\section{Work, heat, first law}

\paragraph{Sign convention. }Contra to engineering, chemical science adopts
a sign convention that is \textit{additive} in nature. $q$ represents the heat
absorbed \textit{by} the system, and $w$ represents work done \textit{on} the
system.


\begin{definition}[First Law of Thermodynamics]
The internal energy $\Delta U$ changes only when a system takes up or gives off
heat $q$ or work $w$.

  \begin{equation}
    \Delta U = q + w
  \end{equation}
\end{definition}

This is a statement of the equivalence of heat and work. The internal energy of
an isolated system is constant; if $\Delta U$ increases in the system, the
energy decreases in the surroundings.

\section{Internal energy, expansion work}

\subsection{Expansion work} 

\paragraph{Expansion work of a gas at constant external pressure.} Because
external pressure is constant, this equation is for irreversible processes.

\begin{equation}
  \delta w' = p_{ext} \mathrm{d}V
\end{equation}

\begin{justification}
  The force which the system moves against is due to the \textit{external
  pressure}. For a piston with area $A$, the force on the piston is $p_{ext}A$. If the
  piston moves out by a small distance $\mathrm{d}x$.

  \begin{equation*}
  \begin{split}
    \delta w' & = F\mathrm{d}x \\
    & =  p_{ext} A \mathrm{d}x \\
    & = p_{ext} \mathrm{d}V
  \end{split}
\end{equation*}

\end{justification}

When the external pressure is zero, the system does no work as it expands,
because it has nothing to push against. This is a \textbf{free expansion}.

\subsubsection{Reversible changes}


Expansions where $p = p_{ext}$. That is, adjusting hypothetical weights on the
piston as the gas is expanding to ensure the above condition. This is to ensure
that $p_{ext}$ is infinitesimally smaller than $p$ at each step of the process.
\textit{Maximum work} is done when an expansion is reversible.

\subsubsection{Isothermal, reversible expansions}

\paragraph{Work done in an isothermal, reversible expansion. }

\begin{equation}
  w' = nRT\ln{\frac{V_f}{V_i}}
\end{equation}

\paragraph{Entropy change for isothermal expansions. }

\begin{equation}
  \Delta S = nR \ln{\frac{V_f}{V_i}}
\end{equation}

\begin{justification}
  For an isothermal system, $\Delta U = 0$. According to the \textbf{First Law
  of Thermodynamics}, $-w = q$. Thus, we can see that
  \begin{equation*}
      q = nRT\ln{\frac{V_f}{V_i}}.
  \end{equation*} Which can be rearranged to give $\Delta S$. Because $S$ is a
  state function and is independent of the path taken, the expression below is
  valid regardless of the reversibility of the expansion.
\end{justification}

\section{Enthalpy}

\begin{definition}
  Enthalpy is the sum of a system's internal energy and the product of its pressure and
  volume. The heat absorbed or released by closed systems at constant pressure
  equals the change in enthalpy.
  \begin{equation}
   H := U + pV
  \end{equation}
\end{definition}

Though this definition may seem arbitrary, a result of consequence is the heat
absorved or released by closed systems at constant pressure equals the change in
enthalpy of the system.

\begin{justification}
  By definition, $H = U + pV$. As the system undergoes a small change of $\mathrm{d}H$,
  \begin{equation*}
    \begin{split}
      H + \mathrm{d}H & = U + \mathrm{d}U + (p + \mathrm{d}V)(V + \mathrm{d}p) \\
      & = U + \mathrm{d}U + pV + p\mathrm{d}V + V\mathrm{d}p +
      \mathrm{d}p\mathrm{d}V
    \end{split}
  \end{equation*}
  As the product of two infinitesimal equations can be neglected,

  \begin{equation*}
    \begin{split}
      H + \mathrm{d}H & = H + \mathrm{d}U + V\mathrm{d}p + p\mathrm{d}V \\
      \mathrm{d}H & = \mathrm{d}U + V\mathrm{d}p + p\mathrm{d}V
    \end{split}
  \end{equation*}
  
  Which is known as the \textit{complete differential} of the definition of
  enthalpy.

  By the \textbf{First Law of Thermodynamics} $\mathrm{d}U = \mathrm{d}q +
  \mathrm{d}w$, the term $\mathrm{d}w = -p\mathrm{d}V$ can be substituted into
  the complete differential as a system under mechanical equilibrium does only
  expansion work.
  
  \begin{equation*}
    \begin{split}
      \mathrm{d}H & = \mathrm{d}U + V\mathrm{d}p + p\mathrm{d}V \\
      & = \mathrm{d}q - p\mathrm{d}V + V\mathrm{d}p + p\mathrm{d}V \\
      & = \mathrm{d}q + V\mathrm{d}p
    \end{split}
  \end{equation*}

Finally, the condition of \textit{constant pressure} is imposed on the system, which is the case for most chemical changes even in a practical sense. This means that $V\mathrm{d}p = 0$. Which gives,

\begin{equation}
  \mathrm{d}H = \mathrm{d}q_{\mathrm{const. pressure}}
\end{equation}

\end{justification}

From the above justification, we can see why $H$ is a good measure of the energy transferred as heat in the various bond-forming and bond-breaking transformations that we encounter in chemistry, which we often invoke in thermochemical calculations.

\section{Adiabatic changes}

\textbf{Adiabatic processes} occur without transfer of energy as heat.

\section{Thermochemistry}

\section{Calorimetry}

\section{The Second Law of Thermodynamics}

\pagebreak

\section{Entropy and the Clausius inequality}

A key aim of physical chemistry is to seek a physical principle or law which
determines which processes will be spontaneous, and which will not. A tempting
idea is that reactions are spontaneous because the products are \textit{more
stable} than the reactants, which implies that they are lower in energy.
However, we can think of examples where processes are spontaneous but are
\textit{not} exothermic. 

Thus, energy minimization is \textit{not} the criterion for spontaneous
processes. This leads to a necessary discussion on the \textbf{Second Law of
Thermodynamics}.

\subsection{Microsopic view of entropy}

\paragraph{Boltzmann distribution. } The most probable distribution of a
particular system. $W$ represents the number of possible configurations of a
system.

\begin{equation}
  S = k_{B} \ln{W}
\end{equation}

We can consider a few factors that increases $S$ by increasing $W$. In fact, most
macroscopic principles regarding entropy can be rationalized as such.

\begin{itemize}
  \item \textit{Supplying energy as heat:} increases $W$ and hence increases the
    entropy.
  \item \textit{Increasing volume:} increasing the number of energy levels
    available, which increases $W$, which
\end{itemize}

The chief reason behind abandoning these definitions and their consequences for
empirical laws is practical in nature, and helps with understanding and
manipulating their effects on bulk matter. We will return to this statistical
understanding on entropy in \textit{statistical mechanics} at a later time.

\begin{definition}[Second Law of Thermodynamics]
  In a spontaneous process, the entropy of the Universe increases
\end{definition}

\section{Entropy and irreversibility}

We've now discussed both entropy and reversible processes separately, now we
will consider how the disorder of a system affects whether a process is
spontaneous, reversible or entirely impossible.

\section{Fundamental equation, absolute $S$, third law}

\subsection{The Master Equation}

\section{Criteria for spontaneous change}

\section{Gibbs free energy}

\section{Helmholtz free energy}

\section{Multicomponent systems, chemical potential}

\section{Chemical equilibrium}

\section{Temperature, pressure and $K_p$}

\section{Equlibrium: application to drug design}

Our analysis of equilibrium typically involves breaking and forming covalent
bonds. Thus, they apply equally well to bio-organic reactions, such as those
involved in ligand-receptor binding.

\section{Phase equilibria}

\subsection{Properties of pure substances}

\subsection{Clausius-Clapyreon equation}

\subsection{Properties of mixtures}

\subsubsection{Ideal solutions}

\subsubsection{Non-ideal solutions}

\subsection{Colligative properties}

\section{Introduction to statistical mechanics}

\section{The partition function}

\subsection{$q$, large $N$ limit}

\subsection{$Q$, many particles}

\section{Statistical mechanics and discrete energy levels}

\section{Model systems}

\section{Applications: chemical and phase equilibria}

\part{Kinetics}

\section{Introduction to reaction kinetics}

\subsection{Defining reaction rate}

\paragraph{Reaction rate.} For a given reaction $\mathrm{aA} + \mathrm{bB}
\rightarrow \mathrm{cC} + \mathrm{dD}$, the reaction rate is

\begin{equation}
  -\frac{1}{\mathrm{a}} \frac{\mathrm{d[A]}}{\mathrm{d}t} 
\end{equation}

\paragraph{Half-life.} $t_{1/2}$ is the time at which $[\mathrm{A}] =
0.5[\mathrm{A}]_0$.

\section{Complex reactions and mechanisms}

\section{Steady-state and equilibrium approximations}

\section{Chain reactions and explosions}

\section{Temperature, $E_a$, catalysis}

\section{Enzyme catalysis}

\section{Autocatalysis and oscillators}

\part{Electrochemistry}

\section{Introduction to electrochemistry}

A simple example of energy storage is the \textbf{capacitor}. Capacitance is
defined as a constant

\begin{equation}
  C = \frac{Q}{V_c}.
\end{equation}

The current is

\begin{equation}
  I = -\frac{\mathrm{d}Q}{\mathrm{d}t} = -C \frac{\mathrm{d}V_c}{\mathrm{d}t}.
\end{equation}

Capacitors in general, fail as an energy storage device due to their lower
specific energy compared to batteries. However, they do have a number of
compelling properties. Storing power as electrical charge rather than chemical
potential has its benefits: this typically allows near instant charge times, and
very high peak output currents. In addition, they can survive many more (around
by a factor of ~$10^3$!) more charge-discharge cycles than fully-cycled
batteries.

\section{The electrochemical cell}

\section{Potentials, interfaces, electrodes and mass transport}

\section{Heterogeneous electron transfer}

\subsection{Tafel equation}

\subsection{Butler-Volmer equation}

\section{Dynamic electrochemical techniques}

\section{Electrochemical energy storage}

\subsection{Batteries}

\subsection{Fuel cells}

\subsection{Supercapacitors and electrode storage}

\section{Analytical methods in electrochemistry}

\subsection{Electrochemical methods}

\subsubsection{Electrochemical impedance spectroscopy}

\subsubsection{Cyclic voltammetry}

\subsubsection{Catalytic peak current analysis}

\subsubsection{Foot-of-the-wave analysis}

\subsubsection{Peak shift analysis}

\subsection{Complementary spectroscopic methods}

\subsubsection{Spectroelectrochemistry}

\subsubsection{Stopped-flow rapid mixing}

\subsubsection{Transient absorption spectroscopy}

\part{Photochemistry}

\section{Introduction}

\subsection{Quantum yields}
To any well-defined photochemical or photophysical process $X$ we can associate
a quantum yield $\Phi$, which is by definition the ratio between the number of
molecules undergoing that process and the number of absorbed photons.

\end{document}
