%% File              : notes.tex
%% Author            : Wayne Yeo [fishnsotong] <wwzyeo@gmail.com>
%% Date              : 2020-04-12T16:57:32+0800
%% Last Modified Date: 2020-06-25T21:40:38+0800
%% Last Modified By  : Wayne Yeo [fishnsotong@gosigs] <wwzyeo@gmail.com>

\documentclass{article}

\usepackage{amsmath,amssymb,amsthm}
\usepackage[shortlabels]{enumitem}
\usepackage{physics}
\usepackage{titlesec}

\newtheorem{theorem}{Theorem}
\numberwithin{theorem}{section}
\newtheorem*{theorem*}{Theorem}
\newtheorem{corollary}{Corollary}
\numberwithin{corollary}{section}
\newtheorem*{corollary*}{Corollary}
\newtheorem{postulate}{Postulate}
\numberwithin{postulate}{section}
\newtheorem*{postulate*}{Postulate}
\newtheorem{lemma}{Lemma}
\numberwithin{lemma}{section}
\newtheorem*{lemma*}{Lemma}
\newtheorem{definition}{Definition}
\numberwithin{definition}{section}
\newtheorem*{definition*}{Definition}

\titleformat
{\part} % command
[display] % shape
{\bfseries\Large\itshape} % format
{Part \ \thepart} % label
{0.5ex} % sep
{
    \rule{\textwidth}{1pt}
    \vspace{1ex}
    \centering
} % before-code
[
\vspace{-0.5ex}%
\rule{\textwidth}{0.3pt}
] % after-code

\DeclareMathOperator{\E}{\mathbb{E}}

\title{5.60 Thermodynamics and Kinetics}
\author{Wayne Yeo Wei Zhong}

\begin{document}

\maketitle

\begin{abstract}
  This course deals primarily with equilibrium properties of macroscopic
  systems, basic thermodynamics, chemical equilibirum of reactions in gas and
  solution phase, and rates of chemical reactions.
\end{abstract}

\tableofcontents
\newpage

\part{Thermodynamics}

\setcounter{section}{0}
\section{States, equations of state, 0th law}

\subsection{Laws of thermodynamics}

\paragraph{0th law of thermodynamics.} If $A$ and $B$ are in thermal equilibrium
and $B$ and $C$ are in thermal equilibrium, then $A$ and $C$ are in thermal 
equilibrium. \textit{Defines Kelvin scale.}

\paragraph{1st law of thermodynamics.} $U$ is conserved. \textit{You can break
even.}

\begin{equation}
  \mathrm{d}w + \mathrm{d}q = \mathrm{d}U
\end{equation}

\paragraph{2nd law of thermodynamics.} Defines entropy and the direction of
time. \textit{You can break even at 0 K}.

\begin{equation}
  \Delta S = \int \frac{\textnormal{d} q_{\mathrm{rev}}}{T}
\end{equation}

\paragraph{3rd law of thermodynamics.} Defines the absolute scale of $T$. For
pure crystal, $S \rightarrow 0$ as $T \rightarrow 0$. \textit{You can never
reach 0 K}.

\subsection{Definitions}

\subsubsection{Types of systems}
\begin{itemize}
  \item Open systems (free transfer of energy and matter).
  \item Closed systems (free transfer of energy, but not matter).
  \item Isolated systems (no transfer of energy and matter).
\end{itemize}

\subsubsection{Types of properties}
\begin{itemize}
  \item{Intensive properties (temperature, molar volume, molar mass, density)}
  \item{Extensive propertiesi (mass, volume, pressure)}
\end{itemize}

You can usually get an intensive property from an extensive property by dividing
it by the number of moles.

\subsection{Properties of gases}

\paragraph{Ideal gas law. }
\begin{equation}
  pV = nRT
\end{equation}

\paragraph{Van der Waals equation of state.} Extension of the ideal gas equation
to account for intermolecular interactions, where $\overline{V}$ denotes molar
volume. 

\begin{equation}
  \left( p + \frac{a}{\overline{V}^2} \right) (\overline{V} - b) = RT 
\end{equation}

Notice that Equation 4 reduces to $pV = nRT$ when $\overline{V}$ is large.

\section{Work, heat, first law}

\section{Internal energy, expansion work}

\section{Enthalpy}

\begin{equation}
  H := U + pV
\end{equation}


\part{Kinetics}

\section{Introduction to reaction kinetics}

\subsection{Defining reaction rate}

\paragraph{Reaction rate.} For a given reaction $\mathrm{aA} + \mathrm{bB}
\rightarrow \mathrm{cC} + \mathrm{dD}$, the reaction rate is

\begin{equation}
  -\frac{1}{\mathrm{a}} \frac{\mathrm{d[A]}}{\mathrm{d}t} 
\end{equation}

\paragraph{Half-life.} $t_{1/2}$ is the time at which $[\mathrm{A}] =
0.5[\mathrm{A}]_0$.


\end{document}
